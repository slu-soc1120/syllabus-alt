\documentclass[]{book}
\usepackage{lmodern}
\usepackage{amssymb,amsmath}
\usepackage{ifxetex,ifluatex}
\usepackage{fixltx2e} % provides \textsubscript
\ifnum 0\ifxetex 1\fi\ifluatex 1\fi=0 % if pdftex
  \usepackage[T1]{fontenc}
  \usepackage[utf8]{inputenc}
\else % if luatex or xelatex
  \ifxetex
    \usepackage{mathspec}
  \else
    \usepackage{fontspec}
  \fi
  \defaultfontfeatures{Ligatures=TeX,Scale=MatchLowercase}
\fi
% use upquote if available, for straight quotes in verbatim environments
\IfFileExists{upquote.sty}{\usepackage{upquote}}{}
% use microtype if available
\IfFileExists{microtype.sty}{%
\usepackage{microtype}
\UseMicrotypeSet[protrusion]{basicmath} % disable protrusion for tt fonts
}{}
\usepackage{hyperref}
\hypersetup{unicode=true,
            pdftitle={SOC 1120-01: Introduction to Sociology - Diversity \& Health},
            pdfauthor={Christopher Prener, Ph.D.},
            pdfborder={0 0 0},
            breaklinks=true}
\urlstyle{same}  % don't use monospace font for urls
\usepackage{natbib}
\bibliographystyle{apalike}
\usepackage{longtable,booktabs}
\usepackage{graphicx,grffile}
\makeatletter
\def\maxwidth{\ifdim\Gin@nat@width>\linewidth\linewidth\else\Gin@nat@width\fi}
\def\maxheight{\ifdim\Gin@nat@height>\textheight\textheight\else\Gin@nat@height\fi}
\makeatother
% Scale images if necessary, so that they will not overflow the page
% margins by default, and it is still possible to overwrite the defaults
% using explicit options in \includegraphics[width, height, ...]{}
\setkeys{Gin}{width=\maxwidth,height=\maxheight,keepaspectratio}
\IfFileExists{parskip.sty}{%
\usepackage{parskip}
}{% else
\setlength{\parindent}{0pt}
\setlength{\parskip}{6pt plus 2pt minus 1pt}
}
\setlength{\emergencystretch}{3em}  % prevent overfull lines
\providecommand{\tightlist}{%
  \setlength{\itemsep}{0pt}\setlength{\parskip}{0pt}}
\setcounter{secnumdepth}{5}
% Redefines (sub)paragraphs to behave more like sections
\ifx\paragraph\undefined\else
\let\oldparagraph\paragraph
\renewcommand{\paragraph}[1]{\oldparagraph{#1}\mbox{}}
\fi
\ifx\subparagraph\undefined\else
\let\oldsubparagraph\subparagraph
\renewcommand{\subparagraph}[1]{\oldsubparagraph{#1}\mbox{}}
\fi

%%% Use protect on footnotes to avoid problems with footnotes in titles
\let\rmarkdownfootnote\footnote%
\def\footnote{\protect\rmarkdownfootnote}

%%% Change title format to be more compact
\usepackage{titling}

% Create subtitle command for use in maketitle
\providecommand{\subtitle}[1]{
  \posttitle{
    \begin{center}\large#1\end{center}
    }
}

\setlength{\droptitle}{-2em}

  \title{SOC 1120-01: Introduction to Sociology - Diversity \& Health}
    \pretitle{\vspace{\droptitle}\centering\huge}
  \posttitle{\par}
    \author{Christopher Prener, Ph.D.}
    \preauthor{\centering\large\emph}
  \postauthor{\par}
      \predate{\centering\large\emph}
  \postdate{\par}
    \date{2019-08-19}

\usepackage{booktabs}
\usepackage{amsthm}
\makeatletter
\def\thm@space@setup{%
  \thm@preskip=8pt plus 2pt minus 4pt
  \thm@postskip=\thm@preskip
}
\makeatother

% work around for errors related to the undefined shaded* enviornment:
\usepackage{color}
\usepackage{fancyvrb}
\newcommand{\VerbBar}{|}
\newcommand{\VERB}{\Verb[commandchars=\\\{\}]}
\DefineVerbatimEnvironment{Highlighting}{Verbatim}{commandchars=\\\{\}}
% Add ',fontsize=\small' for more characters per line
\usepackage{framed}
\definecolor{shadecolor}{RGB}{248,248,248}
\newenvironment{Shaded}{\begin{snugshade}}{\end{snugshade}}
\newcommand{\KeywordTok}[1]{\textcolor[rgb]{0.13,0.29,0.53}{\textbf{#1}}}
\newcommand{\DataTypeTok}[1]{\textcolor[rgb]{0.13,0.29,0.53}{#1}}
\newcommand{\DecValTok}[1]{\textcolor[rgb]{0.00,0.00,0.81}{#1}}
\newcommand{\BaseNTok}[1]{\textcolor[rgb]{0.00,0.00,0.81}{#1}}
\newcommand{\FloatTok}[1]{\textcolor[rgb]{0.00,0.00,0.81}{#1}}
\newcommand{\ConstantTok}[1]{\textcolor[rgb]{0.00,0.00,0.00}{#1}}
\newcommand{\CharTok}[1]{\textcolor[rgb]{0.31,0.60,0.02}{#1}}
\newcommand{\SpecialCharTok}[1]{\textcolor[rgb]{0.00,0.00,0.00}{#1}}
\newcommand{\StringTok}[1]{\textcolor[rgb]{0.31,0.60,0.02}{#1}}
\newcommand{\VerbatimStringTok}[1]{\textcolor[rgb]{0.31,0.60,0.02}{#1}}
\newcommand{\SpecialStringTok}[1]{\textcolor[rgb]{0.31,0.60,0.02}{#1}}
\newcommand{\ImportTok}[1]{#1}
\newcommand{\CommentTok}[1]{\textcolor[rgb]{0.56,0.35,0.01}{\textit{#1}}}
\newcommand{\DocumentationTok}[1]{\textcolor[rgb]{0.56,0.35,0.01}{\textbf{\textit{#1}}}}
\newcommand{\AnnotationTok}[1]{\textcolor[rgb]{0.56,0.35,0.01}{\textbf{\textit{#1}}}}
\newcommand{\CommentVarTok}[1]{\textcolor[rgb]{0.56,0.35,0.01}{\textbf{\textit{#1}}}}
\newcommand{\OtherTok}[1]{\textcolor[rgb]{0.56,0.35,0.01}{#1}}
\newcommand{\FunctionTok}[1]{\textcolor[rgb]{0.00,0.00,0.00}{#1}}
\newcommand{\VariableTok}[1]{\textcolor[rgb]{0.00,0.00,0.00}{#1}}
\newcommand{\ControlFlowTok}[1]{\textcolor[rgb]{0.13,0.29,0.53}{\textbf{#1}}}
\newcommand{\OperatorTok}[1]{\textcolor[rgb]{0.81,0.36,0.00}{\textbf{#1}}}
\newcommand{\BuiltInTok}[1]{#1}
\newcommand{\ExtensionTok}[1]{#1}
\newcommand{\PreprocessorTok}[1]{\textcolor[rgb]{0.56,0.35,0.01}{\textit{#1}}}
\newcommand{\AttributeTok}[1]{\textcolor[rgb]{0.77,0.63,0.00}{#1}}
\newcommand{\RegionMarkerTok}[1]{#1}
\newcommand{\InformationTok}[1]{\textcolor[rgb]{0.56,0.35,0.01}{\textbf{\textit{#1}}}}
\newcommand{\WarningTok}[1]{\textcolor[rgb]{0.56,0.35,0.01}{\textbf{\textit{#1}}}}
\newcommand{\AlertTok}[1]{\textcolor[rgb]{0.94,0.16,0.16}{#1}}
\newcommand{\ErrorTok}[1]{\textcolor[rgb]{0.64,0.00,0.00}{\textbf{#1}}}
\newcommand{\NormalTok}[1]{#1}

% create callout boxes:
\newenvironment{rmdblock}[1]
  {\begin{shaded*}
  \begin{itemize}
  \renewcommand{\labelitemi}{
    \raisebox{-.7\height}[0pt][0pt]{
      {\setkeys{Gin}{width=3em,keepaspectratio}\includegraphics{images/#1}}
    }
  }
  \item
  }
  {
  \end{itemize}
  \end{shaded*}
  }
\newenvironment{rmdnote}
  {\begin{rmdblock}{note}}
  {\end{rmdblock}}
\newenvironment{rmdtip}
  {\begin{rmdblock}{tip}}
  {\end{rmdblock}}
\newenvironment{rmdwarning}
  {\begin{rmdblock}{warning}}
  {\end{rmdblock}}

% set part and section names:
\usepackage{fancyhdr}
\renewcommand{\chaptername}{Section}
\renewcommand\thesection{\Alph{section}}

\begin{document}
\maketitle

\begin{center}
{\huge Preface and Warning} \\
\end{center}
\vspace{5mm}
This is the hardcopy version of the \textbf{Fall 2019} syllabus.
\vspace{5mm}
\par \noindent This \texttt{.pdf} version of the course syllabus is automatically created as part of the document generation process. It is meant for students who wish to keep a hardcopy of the course policies and planned course schedule. \textbf{Since it is automatically created, it is not optimized for easy use} - readers may notice formatting inconsitencies and stray characters that are a result of the markdown to \LaTeX{} conversion process. The web version (located at \href{https://slu-soc1120.github/syllabus-alt/}{https://slu-soc1120.github/syllabus-alt/}) is meant to be the version of the syllabus used for everyday reference during the semester. As such, this \texttt{.pdf} version will not be updated as the semester progresses should any changes to the course schedule be necessary.

\hypertarget{basics}{%
\chapter*{Basics}\label{basics}}
\addcontentsline{toc}{chapter}{Basics}

\hypertarget{course-meetings}{%
\subsection*{Course Meetings}\label{course-meetings}}
\addcontentsline{toc}{subsection}{Course Meetings}

\emph{When:} Tuesdays and Thursdays, 9:30am to 10:45am

\emph{Where:} 240 Cook Hall

\hypertarget{course-website}{%
\subsection*{Course Website}\label{course-website}}
\addcontentsline{toc}{subsection}{Course Website}

\url{https://slu-soc1120.github.io}

\hypertarget{course-materials}{%
\subsection*{Course Materials}\label{course-materials}}
\addcontentsline{toc}{subsection}{Course Materials}

\url{https://classroom.google.com} (you will need an invitation from Chris)

\hypertarget{chriss-information}{%
\subsection*{Chris's Information}\label{chriss-information}}
\addcontentsline{toc}{subsection}{Chris's Information}

\emph{Office:} 1918 Morrissey Hall

\emph{Email:} \href{mailto:chris.prener@slu.edu}{\nolinkurl{chris.prener@slu.edu}}

\emph{Office Hours:} Tuesdays, 11am-noon

\hypertarget{hard-copy-syllabus}{%
\section*{Hard-copy Syllabus}\label{hard-copy-syllabus}}
\addcontentsline{toc}{section}{Hard-copy Syllabus}

If you would like to keep a record of the syllabus, there is a \texttt{.pdf} download button () in the top toolbar.This document will contain a ``snapshot'' of the course policies and planned schedule as of the beginning of the semester but will not be subsequently updated. See the ``Preface and Warning'' on page 2 of the \texttt{.pdf} for additional details.

\hypertarget{change-log}{%
\section*{Change Log}\label{change-log}}
\addcontentsline{toc}{section}{Change Log}

\begin{itemize}
\tightlist
\item
  August 9, 2019 - Initial Spring 2019 release
\item
  August 15, 2019 - Remove Sociological Experience assignment
\item
  August 16, 2019 - Update langauge on \href{/syllabus-alt/student-support.html}{academic accomodations}, \href{/syllabus-alt/academic-honesty.html}{academic honesty}, and \href{/syllabus-alt/harassment-and-title-ix.html}{Title IX} to match latest versions of University syllabi statements
\item
  August 17, 2019 - Fix broken links
\item
  August 19, 2019 - Add link to office hours note in success section
\end{itemize}

\hypertarget{license}{%
\section*{License}\label{license}}
\addcontentsline{toc}{section}{License}

Copyright © 2016-2019 \href{https://chris-prener.github.io}{Christopher G. Prener}

This work is licensed under a Creative Commons Attribution-ShareAlike 4.0 International License.

\hypertarget{part-syllabus}{%
\part{Syllabus}\label{part-syllabus}}

\hypertarget{course-introduction}{%
\chapter{Course Introduction}\label{course-introduction}}

\begin{quote}
The function of sociology, as of every science, is to reveal that which is hidden.
\end{quote}

\textbf{Pierre Bourdieu (1996)}

This course will survey the field of sociology, stressing important ideas, methods, and results. We focus on health to illustrate the application of sociological ideas. The survey is designed to develop analytic thinking skills. Weekly readings from a text will be supplemented with articles and chapters illustrating topical issues and exercises on the skills and craft of the social sciences.

\hypertarget{course-objectives}{%
\section{Course Objectives}\label{course-objectives}}

This course introduces the distinct sociological skills through the lens of health and illness, including:

\begin{enumerate}
\def\labelenumi{\arabic{enumi}.}
\item
  The ability to recognize and examine social phenomena from multiple perspectives.
\item
  The recognition of what constitutes fact based arguments from appropriately designed information gathering.
\item
  The ability to understand the sources of attitudes and behaviors from cultures and structures and how they impact the quality of life of different groups in society.
\item
  The ability to reflect on the diversity around us and to act in a moral and just manner as citizens of the world.
\item
  Developing skills in independent thinking, aesthetic awareness, moral and/or ethical system of values, welcoming diversity, and committing to the value of life-long learning.
\end{enumerate}

\hypertarget{cultural-diversity-core-requirement}{%
\section{Cultural Diversity Core Requirement}\label{cultural-diversity-core-requirement}}

This course fulfills the College of Arts and Sciences core requirement for Cultural Diversity in the United States. The Cultural Diversity in the United States requirement is designed to help students gain a better understanding of the cultural groups in the United States and their interactions. Students who complete a Cultural Diversity course in this category will gain a substantial subset of the following skills:

\begin{enumerate}
\def\labelenumi{\arabic{enumi}.}
\item
  Analyze and evaluate how various underrepresented social groups confront inequality and claim a just place in society.
\item
  Examine how conflict and cooperation between social groups shapes U.S. society and culture.
\item
  Identify how individual and institutional forms of discrimination impact leaders, communities and community building through the examination of such factors as race, ethnicity, gender, religion, economic class, age, physical and mental capability, and sexual orientation.
\item
  Evaluate how their personal life experiences and choices fit within the larger mosaic of U.S. society by confronting and critically analyzing their own values and assumptions about individuals and groups from different cultural contexts.
\item
  Understand how questions of diversity intersect with moral and political questions of justice and equality.
\end{enumerate}

\hypertarget{google-classroom}{%
\section{Google Classroom}\label{google-classroom}}

\textbf{\href{https://classroom.google.com}{Google Classroom}} is a learning management system similar to Blackboard. There are two main areas - the \texttt{Stream} and the \texttt{Coursework} tabs. The \texttt{Stream} contains posts for announcements and assignments. Additions to the \texttt{Stream} should be emailed to your student e-mail account automatically. This will be my primary means for communicating with the class as a whole, and will be the venue where lecture slides and notes are made available. Assignments posted to the \texttt{Coursework} allow you to submit work for the course. Please see the section on ``\protect\hyperlink{google-classroom-submissions}{Google Classroom Submissions}'' for details on assignment submission.

You will need an invitation to \textbf{Google Classroom} from Chris. Invitations will be sent to all enrolled students before the first class. If you enroll after the first day of class, please let Chris know that you will need an invitation. Invitations will be sent to your SLU associated Google account, which consists of your \emph{computer log in} (e.g. \texttt{doej} for Jane Doe) entered as an email - \texttt{doej@slu.edu}. The password will be the same as your password used to log in to mySLU and SLU computers. Using another Google account for this course is not permitted.

\hypertarget{readings}{%
\section{Readings}\label{readings}}

There are two books required for this course. Each book has been selected to correspond with one or more of the course objectives. The books are:

\begin{enumerate}
\def\labelenumi{\arabic{enumi}.}
\item
  Abraham, Laurie K. 2019. \emph{Mama Might Be Better Off Dead: The Failure of Health Care in Urban America}. Chicago, IL: The University of Chicago Press. ISBN-13: 978-0226623702; List Price: \$20.00; e-book versions available.
\item
  Andersen, Margaret, Howard F. Taylor, and Kim A. Logio. 2016. \emph{Sociology: The Essentials}. 9\textsuperscript{th} edition. Independence, KY: Cengage. ISBN-13: 978-1305503083; List Price: \$202.95; e-book versions available.
\end{enumerate}

I do not require students to buy physical copies of texts. You are free to select a means for accessing these texts that meets your budget and learning style. If ebook editions (e.g.~Kindle, iBooks, pdf, etc) of texts are available, they are acceptable for this course. All texts should be obtained in the edition noted above.

All readings are listed on the \href{/syllabus-alt/lecture-schedule.html}{\textbf{Reading List}} and should be completed before the course meeting on the week in which they are assigned (unless otherwise noted). Full text versions of most readings not found in the books assigned for the course can be obtained using the library's \href{http://eres.slu.edu/eres/coursepass.aspx?cid=4443}{Electronic Reserves} system. The password for the Electric Reserves will be posted on \textbf{\href{https://classroom.google.com}{Google Classroom}}.

Many of the readings posted on \href{http://eres.slu.edu/eres/coursepass.aspx?cid=4443}{Electronic Reserves} are peer reviewed journal articles. This means that they are written with an aim to contribute to scientific debates. Their primary audiences are typically health care providers, professors, and graduate students. They are therefore sometimes \emph{difficult} to read. Give yourself time - I don't expect each student to fully understanding the intricacies of each article (especially the statistics included in some), but I do expect you to walk away with a general sense of the argument and evidence presented.

\hypertarget{course-policies}{%
\chapter{Course Policies}\label{course-policies}}

My priority is that class periods are productive learning experiences for all students. In order to foster this type of productive environment, I ask students to follow a few general policies and expectations:\footnote{These general expectations were adopted from language originally used by Dr.~Shelley Kimmelberg.}

\begin{enumerate}
\def\labelenumi{\arabic{enumi}.}
\tightlist
\item
  Work each week to contribute to a positive, supportive, welcoming, and compassionate class environment.
\item
  Arrive to class on time and stay for the entire class period.
\item
  Silence \emph{all} electronic devices before entering the classroom.
\item
  Do not engage in side conversations. This is disrespectful to the speaker (whether me or a classmate), and can affect the ability of others in the class to learn.
\item
  Be respectful of your fellow classmates. Do not interrupt when someone is speaking, monopolize the conversation, or belittle the ideas or opinions of others.
\item
  Complete the assigned readings for each class in advance, and come prepared with discussion points and questions.
\end{enumerate}

The following sections contain additional details about specific course policies related to attendance, participation, electronic device use, student support, academic honesty, and Title IX.

\hypertarget{attendance-and-participation}{%
\section{Attendance and Participation}\label{attendance-and-participation}}

Attendance and participation are important components of this course. Your expected to attend all class sessions and to arrive before the beginning of class. If you cannot attend class or arrive on time because of a personal illness, a family issue, jury duty, an athletic match, or a religious observance, you must contact me \textbf{beforehand} to let me know. I may ask for more information, such as a note from a physician, a travel letter from Athletics, or other documentation for absences.

A penalty will not be applied to your first two unexcused absences or late arrivals. Any unexcused absences or late arrivals beyond those two will result in no credit (for an absences) or only partial credit (for a late arrival) being earned for that day's participation grade.

Attendance check-ins will be collected either in-person or through a simple web-form. Students will need a \href{https://en.wikipedia.org/wiki/QR_code}{QR code} reader application installed on their smartphone to check-in. These web forms are \textbf{time stamped}, so if you sign in 3 minutes after the beginning of class or later, you will be marked as `late' in the attendance database. This is done automatically by my gradebook, so please see me if you have a concern about how this works or, more generally, if you have a commitment that regularly prevents you from arriving to class on time.

If you do not own a smartphone, you will be able to sign-in with me before class. You should note that attendance check-ins are covered by the course's Academic Honesty policy. Sharing the check-in form with another student or signing in on their behalf are both violations of this policy.

Making up missed classes is your responsibility, including obtaining notes from a classmate. I do post slides on \href{https://classroom.google.com}{Google Classroom}, but my slides are intended only to serve as references. I do not design slides to serve as a stand-in for not attending class - they are designed to make sense in the context of the lecture as it is delivered. All lecture slides will be posted on Google Classroom before class begins along with relevant notes for that lecture. Please note that lectures and discussions cannot be recorded by any means (e.g.~audio or video recordings, or photographs) without my permission.

\hypertarget{communication}{%
\section{Communication}\label{communication}}

Email is my preferred method of communication. I dedicate time to email responses each workday, meaning that my response time is typically within 24 hours during the workweek. If you have not received a response from me after 48 hours (or by end of business on Monday if you emailed me over the weekend), please feel free to follow-up with me.

Please use your SLU email account when emailing me. All messages regarding course updates, assignments, and changes to the class schedule including cancellations will be sent to your SLU email account. It is therefore imperative that you check your SLU email account regularly.

Please also ensure that all concerns or questions about your standing in the course are directed to me immediately. Inquires from parents, SLU staff members, and others will not be honored.

\hypertarget{electronic-devices}{%
\section{Electronic Devices}\label{electronic-devices}}

During class periods, students are asked to refrain from using electronic devices (including cell phones) for activities not directly related to the course. For this class, I expect students to limit their use of electronic devices to accessing course readings, notes, and other course materials.

There is evidence that using electronic devices during lectures results in decreased retention of course content (\href{https://link.springer.com/article/10.1007/BF02940852}{Hembrooke and Gay 2003}) and lower overall course performance (\href{https://www.sciencedirect.com/science/article/pii/S0360131506001436}{Fried 2008}). Students who are not using a laptop but are in direct view of another student's laptop also have decreased performance in courses (\href{https://www.sciencedirect.com/science/article/pii/S0360131512002254}{Sana et al.~2013}). Conversely, students who take notes the ``old fashioned way'' have better performance on tests compared to students who take notes on laptops (\href{http://journals.sagepub.com/doi/abs/10.1177/0956797614524581}{Mueller and Oppenheimer 2014}).

I therefore ask students to be conscious of how they are using their devices, the ways such use impacts their own learning, and the effect that it may have on others around them. I reserve the right to alter this policy if electronic device use becomes problematic during the semester.

\hypertarget{student-support}{%
\section{Student Support}\label{student-support}}

\hypertarget{basic-needs}{%
\subsection{Basic Needs}\label{basic-needs}}

If you have difficulty affording groceries or accessing sufficient food to eat every day, or lack a safe and stable place to live, you are urged to contact the \href{https://www.slu.edu/student-development/dean-of-students/index.php}{Dean of Students} for support. Likewise if you have concerns about your mental or physical health needs, or lack access to health care services you require, you should contact either the \href{https://www.slu.edu/student-development/dean-of-students/index.php}{Dean of Students}, \href{https://www.slu.edu/life-at-slu/student-health/index.php}{Student Health Services}, or the \href{https://www.slu.edu/life-at-slu/university-counseling/index.php}{University Counseling Center}.\footnote{This language is adopted from text written by \href{https://medium.com/@saragoldrickrab/basic-needs-security-and-the-syllabus-d24cc7afe8c9}{Dr.~Sarah Goldrick-Rab}.}

If you feel comfortable doing so, please discuss any concerns you might have with me. Doing so is particularly important if believe your performance in this course might be affected. I will do my best to work with you to come up with a plan for successfully completing the course and, if need be, work with you to identify on-campus resources. I will treat all discussions with discretion, though please be aware that certain situations, including disclosures of \href{/syllabus-alt/harassment-and-title-ix.html}{sexual misconduct} or self harm, must be reported by faculty to the appropriate University office.

\hypertarget{academic-accommodations}{%
\subsection{Academic Accommodations}\label{academic-accommodations}}

If you meet the eligibility requirements for academic accommodations through the \href{https://www.slu.edu/life-at-slu/student-success-center/disability-services/index.php}{Office of Disability Services} (located within the Student Success Center) \emph{and you wish to use them for this class}, you should arrange to discuss your needs with me after the first class. All discussions of this nature are treated confidentially, and I will make every effort to work with you to come up with a plan for successfully completing the course requirements.

Please note that I will not provide accommodations to students who are not working with Disability Services, and that I cannot retroactively alter assignments or grades if they have already been completed. This follows the University policies on disability accomodations:

\begin{quote}
Students with a documented disability who wish to request academic accommodations must formally register their disability with the University. Once successfully registered, students also must notify their course instructor that they wish to use their approved accommodations in the course.
\end{quote}

\begin{quote}
Please contact Disability Services to schedule an appointment to discuss accommodation requests and eligibility requirements. Most students on the St.~Louis campus will contact Disability Services, located in the Student Success Center and available by email at \href{mailto:Disability_services@slu.edu}{\nolinkurl{Disability\_services@slu.edu}} or by phone at 314-977-3484. Once approved, information about a student's eligibility for academic accommodations will be shared with course instructors by email from Disability Services and within the instructor's official course roster. Students who do not have a documented disability but who think they may have one also are encouraged to contact to Disability Services. Confidentiality will be observed in all inquiries.
\end{quote}

\hypertarget{writing-services}{%
\subsection{Writing Services}\label{writing-services}}

I also encourage you to take advantage of the \href{https://www.slu.edu/life-at-slu/student-success-center/academic-support/university-writing-services/index.php}{University Writing Services (UWS) program}. Getting feedback benefits writers at all skill levels and the quality of your writing will be reflected in assignment grades. The UWS has trained writing consultants who can help you improve the quality of your written work. UWS's consultants are available to address everything from brainstorming and developing ideas to crafting strong sentences and documenting sources.

\hypertarget{academic-honesty}{%
\section{Academic Honesty}\label{academic-honesty}}

All students should familiarize themselves with \href{https://www.slu.edu/provost/policies/academic-and-course/policy_academic-integrity_6-26-2015.pdf}{Saint Louis University's policies} the the \href{https://www.slu.edu/arts-and-sciences/student-resources/academic-honesty.php}{College of Arts and Sciences policies} concerning cheating, plagiarism, and other academically dishonest practices:

\begin{quote}
Academic integrity is honest, truthful and responsible conduct in all academic endeavors. The mission of Saint Louis University is ``the pursuit of truth for the greater glory of God and for the service of humanity.'' Accordingly, all acts of falsehood demean and compromise the corporate endeavors of teaching, research, health care, and community service through which SLU fulfills its mission. The University strives to prepare students for lives of personal and professional integrity, and therefore regards all breaches of academic integrity as matters of serious concern.
\end{quote}

Any work that is taken from another student, copied from printed material, or copied the internet without proper citation is expressly prohibited, and will be addressed by the instructor.

All relevant assignments should include in-text citations and references formatted using the \href{https://owl.english.purdue.edu/owl/resource/583/1/}{American Sociological Association (ASA)} style guidelines. Any student who is found to have been academically dishonest in their work risks failing both the assignment and this course.

\hypertarget{harassment-and-title-ix}{%
\section{Harassment and Title IX}\label{harassment-and-title-ix}}

While I have every expectation that each member of the Saint Louis University community is capable and willing to create a positive coursework experience, I fully recognize that there may be instances where students fall short of that expectation. Students should generally be aware that:

\begin{quote}
Saint Louis University prohibits harassment because of sex, race, color, religion, national origin, ancestry, disability, age, sexual orientation, marital status, military status, veteran status, gender expression/identity, genetic information, pregnancy, or any other characteristics protected by law.
\end{quote}

All students should also familiarize themselves with \href{http://www.slu.edu/general-counsel/institutional-equity-diversity/}{Saint Louis University's polices} on bias, discrimination, harassment, and sexual misconduct. In particular, they should be aware of policies on \href{http://www.slu.edu/general-counsel/institutional-equity-diversity/harassment.php}{harassment} and \href{https://www.slu.edu/about/safety/sexual-assault-resources.php}{sexual misconduct}:

\begin{quote}
Saint Louis University and its faculty are committed to supporting our students and seeking an environment that is free of bias, discrimination, and harassment. If you have encountered any form of sexual misconduct (e.g., sexual assault, sexual harassment, stalking, domestic or dating violence), we encourage you to report this to the University. If you speak with a faculty member about an incident that involves a Title IX matter, that faculty member must notify SLU's Title IX coordinator (or that person's equivalent on your campus) and share the basic facts of your experience. This is true even if you ask the faculty member not to disclose the incident. The Title IX contact will then be available to assist you in understanding all of your options and in connecting you with all possible resources on and off campus.
\end{quote}

\begin{quote}
For most students on the St.~Louis campus, the appropriate contact is Anna R. Kratky (DuBourg Hall, Room 36; \url{anna.kratky@slu.edu}; 314-977-3886). If you wish to speak with a confidential source, you may contact the counselors at the University Counseling Center at 314-977-TALK. To view SLU's sexual misconduct policy, and for resources, please visit the following web addresses: \url{https://www.slu.edu/here4you} and \url{https://www.slu.edu/general-counsel}.
\end{quote}

Instances of abusive, harassing, or otherwise unacceptable behavior should be reported either directly to the instructor or to the University Administration. Consistent with the above policies, I will forward all reports of inappropriate conduct to the Title IX Coordinator's office or to the Office of Diversity and Affirmative Action. Please be aware that University policies may require me to forward information about the identity of any students connected to the disclosure.

Please also be aware that communications over various online services, including (but not limited to) Google Apps and Google Classroom, are covered by this policy.

\hypertarget{success-in-this-course}{%
\chapter{Success in this Course}\label{success-in-this-course}}

Students often ask me how to do well in various aspects of the course, and so this section features my \emph{suggestions} for a successful semester. \emph{These observations are provided with no warranty} - following them does not guarantee any particular outcome. You could do everything in here and still do poorly in the course, and conversely you could ignore much of what is in discussed in the links below and still do well. However, \emph{most} of the students who are successful in this course will follow \emph{most} of these ideas consistently.

These tips are an effort to illuminate what sociologists refer to as the ``\href{https://books.google.com/books?hl=en\&lr=\&id=5r-TAgAAQBAJ\&oi=fnd\&pg=PP1\&dq=hidden+curriculum\#v=onepage\&q=hidden\%20curriculum\&f=false}{hidden curriculum}'' of higher education - there are things you need to do to be successful, but they are often unstated or not clearly communicated.

Since I give this advice out in multiple classes, the documents themselves are stored on my \href{https://chris-prener.github.io}{personal website}:

\begin{enumerate}
\def\labelenumi{\arabic{enumi}.}
\tightlist
\item
  \href{https://chris-prener.github.io/resources/little-things/}{Doing the Little Things Right}
\item
  \href{https://chris-prener.github.io/resources/office-hours/}{Come to Office Hours!}
\item
  \href{https://chris-prener.github.io/resources/exams/}{Studying for Exams}
\item
  \href{https://chris-prener.github.io/resources/writing/}{Writing in the Social Sciences}
\item
  \href{https://chris-prener.github.io/resources/letters/}{Letters of Recommendation} - Pay particular to the section titled ``\href{https://chris-prener.github.io/resources/letters/\#if-i-say-no}{If I Say No}'' if you may want an Instructor Evaluation for medical school - I prioritize letter writing for students who I get to know outside of the classroom.
\end{enumerate}

If there are other topics you have questions about, please let me know. These documents are a work in progress.

\hypertarget{assignments-and-grading}{%
\chapter{Assignments and Grading}\label{assignments-and-grading}}

This section provides general details on the different types of assignments for this course. It also contains policies for submitting work, receiving feedback, and late work.

\hypertarget{assignments}{%
\section{Assignments}\label{assignments}}

Your grade for this course will consist of a number of different assignments on which points may be earned. Each category of assignment is described below.

\hypertarget{attendance-and-participation-1} of your final grade
\end{rmdtip}

As discussed above, both attendance and participation are important aspects of this class. The class participation grade will be based on (a) attendance, (b) level of engagement during class, and (c) class ``entry'' and ``exit tickets''.

Each of these elements is assigned a point value and assessed using a scale that awards full, partial, or no credit. Not attending class or completing an ``entry'' or ``exit'' ticket will result in no credit being earned for that element on a given day. Disengagement during class may result in partial or no credit being earned. Late arrivals will result in only partial credit earned for that element on a given day.

Your participation grade will be split, with 25 points (5\% of your final grade) for the first half of the semester (through Week 8) and another 25 points (5\%) for the second half (Weeks 9 to 17). Since the number of points awarded for participation are variable, the total number of points earned for each half will be \textbf{weighted} so that it is converted to a 0 to 25 scale. I provide the final number of points earned for each half of the course. If you would like a more detailed breakdown of your participation grade and/or attendance record, please reach out and I will happily provide one.

``Entry'' and ``exit'' tickets will be collected via \textbf{\href{https://classroom.google.com}{Google Classroom}}. These will only be graded for students who present in class on the day that they were collected. Students without access to a smartphone or laptop should submit their tickets as soon as possible after class on the days the tickets are collected. Beginning and end of the semester activities, like the `Student Information Sheet' and the `Speaking Up In Class Survey', are counted as ``entry'' tickets.

\hypertarget{qhq-papers} of your final grade
\end{rmdtip}

Each student will write a QHQ reflection paper on three chapters (one chapter per discussion period) of \emph{Mama Might Be Better Off Dead} (Abraham 1993). These reflection papers will integrate previous lecture material and readings to understand the cycle of events described in the book. Additional details and a grading rubric are available on \textbf{\href{https://classroom.google.com}{Google Classroom}} under the `Classwork' tab. QHQ Paper 1 will be due on \textbf{Thursday, October 10\textsuperscript{th}}, QHQ Paper 2 will be due on \textbf{Thursday, November 14\textsuperscript{th}}, and QHQ Paper 3 will be due on \textbf{Tuesday, December 3\textsuperscript{rd}}. QHQ Paper 1 is ungraded. If it is not turned in or not taking seriously, however, a deduction of 3\% will be applied to your final grade. QHQ Papers 2 and 3 each count for 15\% of your final grade.

\hypertarget{exams} of your final grade
\end{rmdtip}

Three non-cumulative multiple choice exams will be given throughout the semester. Each exam will cover the breadth of the material in the course, including readings, lectures, and videos. Exam 1 will be given on \textbf{Tuesday, September 24\textsuperscript{th}}, Exam 2 will be given on \textbf{Thursday, October 31\textsuperscript{st}}, and Exam 3 will be given during our class's schedule final exam period on \textbf{Tuesday, December 17\textsuperscript{th}}. Please note that Exam 3 will be held at a slightly different time (from 8am to 9:50am, though its format will be the same as the priort exams and most students should not need the full time period). Each exam is worth 20\% of your final grade.

\hypertarget{submission-and-late-work}{%
\section{Submission and Late Work}\label{submission-and-late-work}}

\hypertarget{google-classroom-submissions}{%
\subsection{Google Classroom Submissions}\label{google-classroom-submissions}}

All assignments must be submitted via \textbf{\href{https://classroom.google.com}{Google Classroom}} \emph{as a Google Doc file} unless otherwise noted in the assignment instructions. \textbf{\href{https://classroom.google.com}{Google Classroom}} will have submission portals for each assignment in the \texttt{Coursework} tab. Create your file in Google Docs and then use the submission portal to submit the file. Do not upload a \texttt{.pdf}, \texttt{.docx}, \texttt{.doc}, or other file format - you must submit a \textbf{Google Doc file}. Once submitted, you will not be able to edit the file again until it is returned with feedback and a grade. Feedback will be returned to students via comments embedded in each Google Doc.

The Google Doc submission policy is in place because it facilitates clear, easy grading that can be turned around to you quickly. Submitting assignments in ways that deviate from this policy will result in a late grade (see below) being applied in the first instance and a zero grade for each subsequent instance. If you have questions about how to use Google Docs or Google Classroom, or believe there has been an issue with your submission, please reach out to me \textbf{before} the submission deadline.

\hypertarget{late-work}{%
\subsection{Late Work}\label{late-work}}

Once the due date has passed, any assignments shared will be treated as late. Be advised that Google time-stamps submissions, so that even being a few seconds over the due date and time will result in your assignment being marked late. Like arriving late to class, this happens automatically, so please let me know as soon as possible \textbf{before} class if you have a concern about a potentially late submission.

Assignments shared within 24-hours of the due date will have 15\% deducted from the grade. I will deduct 15\% per day for the next two 24-hour periods that assignments are late; after 72-hours, I will not accept late work. If you cannot submit work on time because of a personal illness, a family issue, jury duty, an athletic match, or a religious observance, you must contact me beforehand to discuss alternate submission of work. Internet or computer issues are not grounds for missed deadlines. I may ask for more information, such as a note from a physician, a travel letter from Athletics, or other documentation for absences.

\hypertarget{missed-exams}{%
\subsection{Missed Exams}\label{missed-exams}}

If you cannot attend an exam due to a personal illness, a family issue, jury duty, an athletic match, or a religious observance, you must contact me \textbf{beforehand} to discuss alternate exam scheduling. I may ask for more information, such as a note from a physician, a travel letter from Athletics, or other documentation for absences. Unless there is an extreme situation, such as emergency hospitalization, I do not allow students to make up exams if I have not been notified prior to the original exam date.

Since the final exam schedule is not yet available, please do not book travel home at the end of the semester until you know the date of the third exam - I cannot accommodate students who request an alternative exam date due to travel conflicts.

\hypertarget{extra-credit}{%
\section{Extra Credit}\label{extra-credit}}

From time to time I may offer extra credit to be applied to your final grade. I will only offer extra credit if it is open to the entire class (typically for something like attending a lecture or event on-campus). If I offer extra credit, I will generally require you to submit a short written summary of the activity within a week of the event to obtain the credit. Papers should be submitted via \textbf{\href{https://classroom.google.com}{Google Classroom}}. When offered, extra credit opportunities cannot be made-up or substituted if you are unable to attend the event.

\hypertarget{grading}{%
\section{Grading}\label{grading}}

I use a point system for calculating grades. The following table gives the weighting and final point totals for all assignments for this course:

\begin{table}[t]

\caption{\label{tab:unnamed-chunk-4}SOC 1120 Points Breakdown}
\centering
\begin{tabular}{lllll}
\toprule
Assignment & Points & Quantity & Total & Percent\\
\midrule
Participation & 25 pts & x2 & 50 pts & 10\%\\
QHQs & 75 pts & x2 & 150 pts & 30\%\\
Exams & 100 pts & x3 & 300 pts & 60\%\\
\bottomrule
\end{tabular}
\end{table}

All feedback will include grades that represent number of points earned. If you want to know your percentage on a particular assignment, divide the number of points earned by the number of points possible and then multiply it by 100.

Some of the provided rubrics on \textbf{\href{https://classroom.google.com}{Google Classroom}} result in final points for assignments that include decimals. In the event of non-standard decimals (those other than .25, .5, or .75), I will round your grade up to the next standard decimal value (e.g. .25, .5, or .75).

\hypertarget{conflicting-or-incorrect-grades}{%
\subsection{Conflicting or Incorrect Grades}\label{conflicting-or-incorrect-grades}}

Assignment grades will generally be given both as a comment in the Google Doc itself and on Google Classroom. If you notice a discrepancy between the grade you received in the feedback and what appears on Google Classroom, please let me know as soon as possible. I will default to taking the higher of the two grades as the official grade.

Since exams and scantrons are handed back in class, I will only consider possible scantron scoring errors that I am made aware of at the end of class on the day that exams are returned. Please make sure to discuss any concerns you have about your scantron with me that day. As with other assignments, if there is a discrepancy between the grade noted on your exam and what is entered on Google Classroom, I will default to taking the higher of the two grades as the official grade.

\hypertarget{letter-grades}{%
\subsection{Letter Grades}\label{letter-grades}}

letter grades will be calculated by taking the sum of all points earned and dividing it by the total number of points possible. This will be multiplied by 100 and then converted to a letter grade using the following table:

\begin{table}
\caption{\label{tab:unnamed-chunk-5}Course Grading Scale}

\centering
\begin{tabular}{lll}
\toprule
GPA & Letter & Percent\\
\midrule
4.0 & A & 93.0\% - 100\%\\
3.7 & A- & 90.0\% - 92.9\%\\
3.3 & B+ & 87.0\% - 89.9\%\\
3.0 & B & 83.0\% - 86.9\%\\
2.7 & B- & 80.0\% - 82.9\%\\
\bottomrule
\end{tabular}
\centering
\begin{tabular}{lll}
\toprule
GPA & Letter & Percent\\
\midrule
2.3 & C+ & 77.0\% - 79.9\%\\
2.0 & C & 73.0\% - 76.9\%\\
1.7 & C- & 70.0\% - 72.9\%\\
1.0 & D & 63.0\% - 69.9\%\\
0.0 & F & < 63.0\%\\
\bottomrule
\end{tabular}
\end{table}

For this course, midterm grades will be calculated out of a total of 150 possible points (first half participation and the first exam), a ``week 16'' grade will be calculated out of 400 possible points (all grades except the third exam), and the final grade will be calculated out of 500 possible points. Extra credit will be factored into the ``week 16'' and final grades. All letter grades will be disseminated via Google Classroom, and midterm as well as final grades will also be made available through Blackboard.

\hypertarget{revisions-and-incompletes}{%
\subsection{Revisions and Incompletes}\label{revisions-and-incompletes}}

No chances will be given for revisions of poor grades. Incomplete grades will be given upon request only if you have a ``C'' average and have completed at least two-thirds of the possible points (330 points). You should note that incomplete grades must be rectified by the specified deadline or they convert to an ``F''.

\hypertarget{part-reading-list}{%
\part{Reading List}\label{part-reading-list}}

\hypertarget{course-schedule}{%
\chapter{Course Schedule}\label{course-schedule}}

The following is a high-level schedule that details the general topic covered by each lecture.

\begin{table}[t]

\caption{\label{tab:unnamed-chunk-1}SOC 1120 Course Overview}
\centering
\begin{tabular}{lll}
\toprule
Week & Tuesday & Topic\\
\midrule
01 & August 27\textasciicircum{}th\textasciicircum{} & Course Introduction; Inequality \& Health\\
02 & September 3\textasciicircum{}rd\textasciicircum{} & *Labor Day* / Theory \& Health\\
03 & September 10\textasciicircum{}th\textasciicircum{} & Culture\\
04 & September 17\textasciicircum{}th\textasciicircum{} & Socialization\\
05 & September 24\textasciicircum{}th\textasciicircum{} & Exam 1 / Social Structure\\
\addlinespace
06 & October 1\textasciicircum{}st\textasciicircum{} & Social Constructionism / Urban Sociology\\
07 & October 8\textasciicircum{}th\textasciicircum{} & Urban Sociology / *Mama*, Part 1\\
08 & October 15\textasciicircum{}th\textasciicircum{} & Crime \& Deviance\\
09 & October 22\textasciicircum{}nd\textasciicircum{} & *Fall Break* / Class \& Stratification\\
10 & October 29\textasciicircum{}th\textasciicircum{} & Stratification \& Health / Exam 2\\
\addlinespace
11 & November 5\textasciicircum{}th\textasciicircum{} & Race \& Ethnicity\\
12 & November 12\textasciicircum{}th\textasciicircum{} & Gender \& Sexuality / *Mama*, Part 2\\
13 & November 19\textasciicircum{}th\textasciicircum{} & The Health Care Experience / Indigenous Americans\\
14 & November 26\textasciicircum{}th\textasciicircum{} & Intersectionality \& Health / *Thanksgiving*\\
15 & December 3\textasciicircum{}rd\textasciicircum{} & *Mama*, Part 3 / Course Conclusion\\
\addlinespace
16 & December 10\textasciicircum{}th\textasciicircum{} & *No Class*\\
17 & December 17\textasciicircum{}th\textasciicircum{} & Exam 3\\
\bottomrule
\end{tabular}
\end{table}

\hypertarget{scheduling-notes}{%
\subsection{Scheduling Notes}\label{scheduling-notes}}

The lecture schedule may change as it depends on the progress of the class. In the event of a cancellation due to weather or another disruption, I may alter the lecture schedule. The web version of this document will be updated to reflect any alterations, but the \texttt{.pdf} version will remain unaltered.

\hypertarget{lecture-schedule}{%
\chapter{Lecture Schedule}\label{lecture-schedule}}

Select a lecture from the menu to see details about topics, readings, and assignments. Additional notes and links to course materials are available through the \href{https://classroom.google.com}{Google Classroom}, which has dedicated pages for each lecture. Links to these pages are included on each lecture's reading list entry.

The primary readings will be referred to with an abbreviation each time they appear in the reading list:

\begin{table}[t]

\caption{\label{tab:unnamed-chunk-1}SOC 1120 Primary Readings}
\centering
\begin{tabular}{ll}
\toprule
Abbreviation & Citation\\
\midrule
Anderson & Andersen, Margaret, Howard F. Taylor, and Kim A. Logio. 2016. *Sociology: The Essentials*. 9\textasciicircum{}th\textasciicircum{} edition. Independence, KY: Cengage.\\
*Mamma* & Abraham, Laurie K. 2019. *Mama Might Be Better Off Dead: The Failure of Health Care in Urban America*. Chicago, IL: The University of Chicago Press.\\
\bottomrule
\end{tabular}
\end{table}

\newpage

\hypertarget{week-01}{%
\section*{Week 01}\label{week-01}}
\addcontentsline{toc}{section}{Week 01}

\hypertarget{lecture-01---tuesday-august-27th}{%
\subsection*{\texorpdfstring{Lecture 01 - Tuesday, August 27\textsuperscript{th}}{Lecture 01 - Tuesday, August 27th}}\label{lecture-01---tuesday-august-27th}}
\addcontentsline{toc}{subsection}{Lecture 01 - Tuesday, August 27\textsuperscript{th}}

\hypertarget{topics}{%
\subsubsection*{Topics}\label{topics}}
\addcontentsline{toc}{subsubsection}{Topics}

\begin{itemize}
\tightlist
\item
  Course Introduction
\end{itemize}

\begin{center}\rule{0.5\linewidth}{\linethickness}\end{center}

\hypertarget{lecture-02---thursday-august-29th}{%
\subsection*{\texorpdfstring{Lecture 02 - Thursday, August 29\textsuperscript{th}}{Lecture 02 - Thursday, August 29th}}\label{lecture-02---thursday-august-29th}}
\addcontentsline{toc}{subsection}{Lecture 02 - Thursday, August 29\textsuperscript{th}}

\hypertarget{topics-1}{%
\subsubsection*{Topics}\label{topics-1}}
\addcontentsline{toc}{subsubsection}{Topics}

\begin{itemize}
\tightlist
\item
  Sociological Theory
\item
  Inequality \& Health
\end{itemize}

\hypertarget{documentary}{%
\subsubsection*{Documentary}\label{documentary}}
\addcontentsline{toc}{subsubsection}{Documentary}

\begin{itemize}
\tightlist
\item
  \emph{Unnatural Causes}, Part 1 - ``In Sickness and in Wealth'' (Pius Library)
\end{itemize}

\hypertarget{readings-1}{%
\subsubsection*{Readings}\label{readings-1}}
\addcontentsline{toc}{subsubsection}{Readings}

\begin{itemize}
\tightlist
\item
  Andersen, Chapter 1 - ``The Sociological Perspective'' (Electronic Reserves)
\item
  Carter, Gregg L. 2009. ``A Primer on Critical Reading.'' Pp. 1-5 in \emph{Empirical Approaches to Sociology: A Collection of Classic and Contemporary Readings}, edited by G.L. Carter. 5\textsuperscript{th} ed. New York, NY: Pearson. (Electronic Reserves)
\end{itemize}

\hypertarget{assignments-due}{%
\subsubsection*{Assignments Due}\label{assignments-due}}
\addcontentsline{toc}{subsubsection}{Assignments Due}

\begin{itemize}
\tightlist
\item
  Entry Ticket - Student Information Sheet (Google Classroom)
\end{itemize}

\newpage

\hypertarget{week-02}{%
\section*{Week 02}\label{week-02}}
\addcontentsline{toc}{section}{Week 02}

\hypertarget{no-class---tuesday-september-3rd}{%
\subsection*{\texorpdfstring{\emph{No Class} - Tuesday, September 3\textsuperscript{rd}}{No Class - Tuesday, September 3rd}}\label{no-class---tuesday-september-3rd}}
\addcontentsline{toc}{subsection}{\emph{No Class} - Tuesday, September 3\textsuperscript{rd}}

\hypertarget{topics-2}{%
\subsubsection*{Topics}\label{topics-2}}
\addcontentsline{toc}{subsubsection}{Topics}

\begin{itemize}
\tightlist
\item
  The Roots of Labor Day
\end{itemize}

\hypertarget{readings-2}{%
\subsubsection*{Readings}\label{readings-2}}
\addcontentsline{toc}{subsubsection}{Readings}

\begin{itemize}
\tightlist
\item
  Kazin, Michael, and Steven J. Ross. 1992. ``America's Labor Day: The Dilemma of a Workers' Celebration.'' \emph{The Journal of American History} 78(4):1294-1323. (Electronic Reserves)
\end{itemize}

\begin{center}\rule{0.5\linewidth}{\linethickness}\end{center}

\hypertarget{lecture-03---thursday-september-5th}{%
\subsection*{\texorpdfstring{Lecture 03 - Thursday, September 5\textsuperscript{th}}{Lecture 03 - Thursday, September 5th}}\label{lecture-03---thursday-september-5th}}
\addcontentsline{toc}{subsection}{Lecture 03 - Thursday, September 5\textsuperscript{th}}

\hypertarget{topics-3}{%
\subsubsection*{Topics}\label{topics-3}}
\addcontentsline{toc}{subsubsection}{Topics}

\begin{itemize}
\tightlist
\item
  Social Science Research - The Sociological Imagination
\item
  Theory and Health - Fundamental Cause Theory
\end{itemize}

\hypertarget{readings-3}{%
\subsubsection*{Readings}\label{readings-3}}
\addcontentsline{toc}{subsubsection}{Readings}

\begin{itemize}
\tightlist
\item
  Phelan, Jo C., Bruce Link, and Parisa Tehranifar. 2010. ``Social Conditions as Fundamental Causes of Health Inequalities: Theory, Evidence, and Policy Implications.'' \emph{Journal of Health and Social Behavior} 51(S):S28-S40. (Electronic Reserves)
\item
  \emph{Mama} - Preface and Introduction (Electronic Reserves)
\end{itemize}

\hypertarget{assignments-due-1}{%
\subsubsection*{Assignments Due}\label{assignments-due-1}}
\addcontentsline{toc}{subsubsection}{Assignments Due}

\begin{itemize}
\tightlist
\item
  Entry Ticket - Reflecting on Labor Day's Roots (Google Classroom)
\end{itemize}

\newpage

\hypertarget{week-03}{%
\section*{Week 03}\label{week-03}}
\addcontentsline{toc}{section}{Week 03}

\hypertarget{lecture-04---tuesday-september-10th}{%
\subsection*{\texorpdfstring{Lecture 04 - Tuesday, September 10\textsuperscript{th}}{Lecture 04 - Tuesday, September 10th}}\label{lecture-04---tuesday-september-10th}}
\addcontentsline{toc}{subsection}{Lecture 04 - Tuesday, September 10\textsuperscript{th}}

\hypertarget{topics-4}{%
\subsubsection*{Topics}\label{topics-4}}
\addcontentsline{toc}{subsubsection}{Topics}

\begin{itemize}
\tightlist
\item
  Social Science Research - Praxis and Method
\item
  What is Culture?
\end{itemize}

\hypertarget{readings-4}{%
\subsubsection*{Readings}\label{readings-4}}
\addcontentsline{toc}{subsubsection}{Readings}

\begin{itemize}
\tightlist
\item
  Anderson, Chapter 2
\item
  Anderson, Chapter 3 - ``Doing Sociological Research'', pp.~57-72

  \begin{itemize}
  \tightlist
  \item
    read up to ``Research Ethics: Is Sociology Value Free?''
  \end{itemize}
\end{itemize}

\begin{center}\rule{0.5\linewidth}{\linethickness}\end{center}

\hypertarget{lecture-05---thursday-september-12th}{%
\subsection*{\texorpdfstring{Lecture 05 - Thursday, September 12\textsuperscript{th}}{Lecture 05 - Thursday, September 12th}}\label{lecture-05---thursday-september-12th}}
\addcontentsline{toc}{subsection}{Lecture 05 - Thursday, September 12\textsuperscript{th}}

\hypertarget{topics-5}{%
\subsubsection*{Topics}\label{topics-5}}
\addcontentsline{toc}{subsubsection}{Topics}

\begin{itemize}
\tightlist
\item
  Social Science Research - Research Ethics
\item
  Culture \& Health
\end{itemize}

\hypertarget{documentary-1}{%
\subsubsection*{Documentary}\label{documentary-1}}
\addcontentsline{toc}{subsubsection}{Documentary}

\begin{itemize}
\tightlist
\item
  \emph{Unnatural Causes}, Part 3 - ``Becoming Americans'' (Pius Library)
\end{itemize}

\hypertarget{readings-5}{%
\subsubsection*{Readings}\label{readings-5}}
\addcontentsline{toc}{subsubsection}{Readings}

\begin{itemize}
\tightlist
\item
  Acevedo-Garcia, Dolores and Lisa M. Bates. 2008. ``Latino Health Paradoxes: Empirical Evidence, Explanations, Future Research, and Implications.'' Pp. 101-113 in \emph{Latinas/os in the United States: Changing the Face of América}, edited by H. Rodríguez, R. Sáenz, and C. Menjívar. New York: Springer. (Electronic Reserves)
\item
  Anderson, Chapter 3 - ``Doing Sociological Research'', pp.~72-74

  \begin{itemize}
  \tightlist
  \item
    read ``Research Ethics: Is Sociology Value Free?''
  \end{itemize}
\end{itemize}

\newpage

\hypertarget{week-04}{%
\section*{Week 04}\label{week-04}}
\addcontentsline{toc}{section}{Week 04}

\hypertarget{lecture-06---tuesday-september-17th}{%
\subsection*{\texorpdfstring{Lecture 06 - Tuesday, September 17\textsuperscript{th}}{Lecture 06 - Tuesday, September 17th}}\label{lecture-06---tuesday-september-17th}}
\addcontentsline{toc}{subsection}{Lecture 06 - Tuesday, September 17\textsuperscript{th}}

\hypertarget{topics-6}{%
\subsubsection*{Topics}\label{topics-6}}
\addcontentsline{toc}{subsubsection}{Topics}

\begin{itemize}
\tightlist
\item
  Nature, Nurture, \& Socialization
\end{itemize}

\hypertarget{readings-6}{%
\subsubsection*{Readings}\label{readings-6}}
\addcontentsline{toc}{subsubsection}{Readings}

\begin{itemize}
\tightlist
\item
  Anderson, Chapter 4 - ``Socialization and the Life Course'', pp.~77-79

  \begin{itemize}
  \tightlist
  \item
    read through the end of ``The Nature-Nurture Controversy''
  \end{itemize}
\item
  Bearman, Peter. 2008. ``Introduction: Exploring Genetics and Social Structure.'' \emph{American Journal of Sociology} 114(S1):v-x. (Electronic Reserves)
\end{itemize}

\begin{center}\rule{0.5\linewidth}{\linethickness}\end{center}

\hypertarget{lecture-07---thursday-september-19th}{%
\subsection*{\texorpdfstring{Lecture 07 - Thursday, September 19\textsuperscript{th}}{Lecture 07 - Thursday, September 19th}}\label{lecture-07---thursday-september-19th}}
\addcontentsline{toc}{subsection}{Lecture 07 - Thursday, September 19\textsuperscript{th}}

\hypertarget{topics-7}{%
\subsubsection*{Topics}\label{topics-7}}
\addcontentsline{toc}{subsubsection}{Topics}

\begin{itemize}
\tightlist
\item
  The Socialization Process
\item
  The Life Course Perspective
\end{itemize}

\hypertarget{readings-7}{%
\subsubsection*{Readings}\label{readings-7}}
\addcontentsline{toc}{subsubsection}{Readings}

\begin{itemize}
\tightlist
\item
  Anderson, Chapter 4 - ``Socialization and the Life Course'', pp.~80-101
\item
  Braveman, Paula and Colleen Barclay. 2009. ``Health Disparities Beginning in Childhood: A Life-Course Perspective.'' \emph{Pediatrics} 123(S3):S163-S175. (Electronic Reserves)
\end{itemize}

\newpage

\hypertarget{week-05}{%
\section*{Week 05}\label{week-05}}
\addcontentsline{toc}{section}{Week 05}

\hypertarget{exam-1---tuesday-september-24th}{%
\subsection*{\texorpdfstring{Exam 1 - Tuesday, September 24\textsuperscript{th}}{Exam 1 - Tuesday, September 24th}}\label{exam-1---tuesday-september-24th}}
\addcontentsline{toc}{subsection}{Exam 1 - Tuesday, September 24\textsuperscript{th}}

\hypertarget{topics-8}{%
\subsubsection*{Topics}\label{topics-8}}
\addcontentsline{toc}{subsubsection}{Topics}

\begin{itemize}
\tightlist
\item
  covers Lectures 1 through 7, all associated readings, and the Labor Day reading
\end{itemize}

\begin{center}\rule{0.5\linewidth}{\linethickness}\end{center}

\hypertarget{lecture-08---thursday-september-26th}{%
\subsection*{\texorpdfstring{Lecture 08 - Thursday, September 26\textsuperscript{th}}{Lecture 08 - Thursday, September 26th}}\label{lecture-08---thursday-september-26th}}
\addcontentsline{toc}{subsection}{Lecture 08 - Thursday, September 26\textsuperscript{th}}

\hypertarget{topics-9}{%
\subsubsection*{Topics}\label{topics-9}}
\addcontentsline{toc}{subsubsection}{Topics}

\begin{itemize}
\tightlist
\item
  Structure \& Structural Inequalities
\end{itemize}

\hypertarget{readings-8}{%
\subsubsection*{Readings}\label{readings-8}}
\addcontentsline{toc}{subsubsection}{Readings}

\begin{itemize}
\tightlist
\item
  Andersen, Chapter 5 - ``Social Structure and Social Interaction''
\end{itemize}

\newpage

\hypertarget{week-06}{%
\section*{Week 06}\label{week-06}}
\addcontentsline{toc}{section}{Week 06}

\hypertarget{lecture-09---tuesday-october-1st}{%
\subsection*{\texorpdfstring{Lecture 09 - Tuesday, October 1\textsuperscript{st}}{Lecture 09 - Tuesday, October 1st}}\label{lecture-09---tuesday-october-1st}}
\addcontentsline{toc}{subsection}{Lecture 09 - Tuesday, October 1\textsuperscript{st}}

\hypertarget{topics-10}{%
\subsubsection*{Topics}\label{topics-10}}
\addcontentsline{toc}{subsubsection}{Topics}

\begin{itemize}
\tightlist
\item
  The Social Construction of Health
\item
  Medicalization
\end{itemize}

\hypertarget{readings-9}{%
\subsubsection*{Readings}\label{readings-9}}
\addcontentsline{toc}{subsubsection}{Readings}

\begin{itemize}
\tightlist
\item
  Conrad, Peter and Kristin K. Barker. 2010. ``The Social Construction of Illness: Key Insights and Policy Implications.'' \emph{Journal of Health and Social Behavior} 51(S):S67-S79. (Electronic Reserves)
\end{itemize}

\begin{center}\rule{0.5\linewidth}{\linethickness}\end{center}

\hypertarget{lecture-10---thursday-october-3rd}{%
\subsection*{\texorpdfstring{Lecture 10 - Thursday, October 3\textsuperscript{rd}}{Lecture 10 - Thursday, October 3rd}}\label{lecture-10---thursday-october-3rd}}
\addcontentsline{toc}{subsection}{Lecture 10 - Thursday, October 3\textsuperscript{rd}}

\hypertarget{topics-11}{%
\subsubsection*{Topics}\label{topics-11}}
\addcontentsline{toc}{subsubsection}{Topics}

\begin{itemize}
\tightlist
\item
  Urban Sociology
\end{itemize}

\hypertarget{readings-10}{%
\subsubsection*{Readings}\label{readings-10}}
\addcontentsline{toc}{subsubsection}{Readings}

\begin{itemize}
\tightlist
\item
  For the Sake of All Project. 2018. \emph{Segregation in St.~Louis: Dismantling the Divide.} St.~Louis, MO: Washington University in St.~Louis and Saint Louis University. (Link)

  \begin{itemize}
  \tightlist
  \item
    Chapter 1 - ``Segregation at the center'', pp.~4-13
  \item
    Chapter 2 - ``St.~Louis: A city of promise, a history of segregation'', pp.~14-25
  \item
    Chapter 5 - ``Segregation in St.~Louis today'', pp.~64-85
  \end{itemize}
\end{itemize}

\newpage

\hypertarget{week-07}{%
\section*{Week 07}\label{week-07}}
\addcontentsline{toc}{section}{Week 07}

\hypertarget{lecture-11---tuesday-october-8th}{%
\subsection*{\texorpdfstring{Lecture 11 - Tuesday, October 8\textsuperscript{th}}{Lecture 11 - Tuesday, October 8th}}\label{lecture-11---tuesday-october-8th}}
\addcontentsline{toc}{subsection}{Lecture 11 - Tuesday, October 8\textsuperscript{th}}

\hypertarget{topics-12}{%
\subsubsection*{Topics}\label{topics-12}}
\addcontentsline{toc}{subsubsection}{Topics}

\begin{itemize}
\tightlist
\item
  Neighborhoods \& Health
\item
  Urban Health Disparities in St.~Louis
\end{itemize}

\hypertarget{readings-11}{%
\subsubsection*{Readings}\label{readings-11}}
\addcontentsline{toc}{subsubsection}{Readings}

\begin{itemize}
\tightlist
\item
  For the Sake of All Project. 2015. \emph{For the Sake of All: A report on the health and well-being of African Americans in St.~Louis and why it matters for everyone.} St.~Louis, MO: Washington University in St.~Louis and Saint Louis University. (Link)

  \begin{itemize}
  \tightlist
  \item
    Chapter 1 - ``Introduction: Why consider economics, education, and health together?'', pp.~10-15
  \item
    Chapter 3 - ``Place matters: Neighborhood resources and health'', pp.~26-33
  \item
    Chapter 5 - ``A health profile of African Americans in St.~Louis'', pp.~46-67
  \end{itemize}
\end{itemize}

\begin{center}\rule{0.5\linewidth}{\linethickness}\end{center}

\hypertarget{qhq-1---thursday-october-10th}{%
\subsection*{\texorpdfstring{QHQ 1 - Thursday, October 10\textsuperscript{th}}{QHQ 1 - Thursday, October 10th}}\label{qhq-1---thursday-october-10th}}
\addcontentsline{toc}{subsection}{QHQ 1 - Thursday, October 10\textsuperscript{th}}

\hypertarget{topics-13}{%
\subsubsection*{Topics}\label{topics-13}}
\addcontentsline{toc}{subsubsection}{Topics}

\begin{itemize}
\tightlist
\item
  Discussion - \emph{Mama Might Be Better Off Dead}, Part 1
\end{itemize}

\hypertarget{readings-12}{%
\subsubsection*{Readings}\label{readings-12}}
\addcontentsline{toc}{subsubsection}{Readings}

\begin{itemize}
\tightlist
\item
  Abraham - Chapters 1 through 5
\end{itemize}

\hypertarget{assignments-due-2}{%
\subsubsection*{Assignments Due}\label{assignments-due-2}}
\addcontentsline{toc}{subsubsection}{Assignments Due}

\begin{itemize}
\tightlist
\item
  QHQ 1 (Google Classroom)

  \begin{itemize}
  \tightlist
  \item
    see the QHQ Group assignments assignments for which chapter you should write about
  \end{itemize}
\end{itemize}

\newpage

\hypertarget{week-08}{%
\section*{Week 08}\label{week-08}}
\addcontentsline{toc}{section}{Week 08}

\hypertarget{lecture-12---tuesday-october-15th}{%
\subsection*{\texorpdfstring{Lecture 12 - Tuesday, October 15\textsuperscript{th}}{Lecture 12 - Tuesday, October 15th}}\label{lecture-12---tuesday-october-15th}}
\addcontentsline{toc}{subsection}{Lecture 12 - Tuesday, October 15\textsuperscript{th}}

\hypertarget{topics-14}{%
\subsubsection*{Topics}\label{topics-14}}
\addcontentsline{toc}{subsubsection}{Topics}

\begin{itemize}
\tightlist
\item
  The Social Construction of Deviance
\item
  Disparities in Crime
\end{itemize}

\hypertarget{readings-13}{%
\subsubsection*{Readings}\label{readings-13}}
\addcontentsline{toc}{subsubsection}{Readings}

\begin{itemize}
\tightlist
\item
  Andersen, Chapter 7 - ``Deviance and Crime'', pp.~147-160

  \begin{itemize}
  \tightlist
  \item
    read through the end of ``Measuring Crime: How Much Is There?''
  \end{itemize}
\end{itemize}

\begin{center}\rule{0.5\linewidth}{\linethickness}\end{center}

\hypertarget{lecture-13---thursday-october-17th}{%
\subsection*{\texorpdfstring{Lecture 13 - Thursday, October 17\textsuperscript{th}}{Lecture 13 - Thursday, October 17th}}\label{lecture-13---thursday-october-17th}}
\addcontentsline{toc}{subsection}{Lecture 13 - Thursday, October 17\textsuperscript{th}}

\hypertarget{topics-15}{%
\subsubsection*{Topics}\label{topics-15}}
\addcontentsline{toc}{subsubsection}{Topics}

\begin{itemize}
\tightlist
\item
  The War on Drugs
\item
  Social Responses to Crime
\item
  The Mark of a Criminal Record
\end{itemize}

\hypertarget{readings-14}{%
\subsubsection*{Readings}\label{readings-14}}
\addcontentsline{toc}{subsubsection}{Readings}

\begin{itemize}
\tightlist
\item
  Anderson, Chapter 7 - ``Deviance and Crime'', pp.~160-167

  \begin{itemize}
  \tightlist
  \item
    read from ``Types of Crime'' onward
  \end{itemize}
\item
  Bourgois, Phillipe. 2008. ``The Mystery of Marijuana: Science and the U.S. War on Drugs.'' Substance Use and Misuse 43: 581-583. (Electronic Reserves)
\end{itemize}

\newpage

\hypertarget{week-09}{%
\section*{Week 09}\label{week-09}}
\addcontentsline{toc}{section}{Week 09}

\hypertarget{no-class---tuesday-october-22nd---fall-break}{%
\subsection*{\texorpdfstring{\emph{No Class} - Tuesday, October 22\textsuperscript{nd} - Fall Break}{No Class - Tuesday, October 22nd - Fall Break}}\label{no-class---tuesday-october-22nd---fall-break}}
\addcontentsline{toc}{subsection}{\emph{No Class} - Tuesday, October 22\textsuperscript{nd} - Fall Break}

\hypertarget{topics-16}{%
\subsubsection*{Topics}\label{topics-16}}
\addcontentsline{toc}{subsubsection}{Topics}

\begin{itemize}
\tightlist
\item
  Michael Brown's Death
\end{itemize}

\hypertarget{readings-15}{%
\subsubsection*{Readings}\label{readings-15}}
\addcontentsline{toc}{subsubsection}{Readings}

\begin{itemize}
\tightlist
\item
  Civil Rights Division. 2015. \emph{Investigation of the Ferguson Police Department}. Washington, DC: U.S. Department of Justice. (Link)

  \begin{itemize}
  \tightlist
  \item
    Part 1 - ``Report Summary'', pp.~1-6
  \item
    Part 2 - ``Background'', pp 6-9
  \item
    Part 3 - ``Ferguson Law Enforcement Efforts Are Focused on Generating Revenue'', pp.~9-15
  \end{itemize}
\end{itemize}

\begin{center}\rule{0.5\linewidth}{\linethickness}\end{center}

\hypertarget{lecture-14---thursday-october-24th}{%
\subsection*{\texorpdfstring{Lecture 14 - Thursday, October 24\textsuperscript{th}}{Lecture 14 - Thursday, October 24th}}\label{lecture-14---thursday-october-24th}}
\addcontentsline{toc}{subsection}{Lecture 14 - Thursday, October 24\textsuperscript{th}}

\hypertarget{topics-17}{%
\subsubsection*{Topics}\label{topics-17}}
\addcontentsline{toc}{subsubsection}{Topics}

\begin{itemize}
\tightlist
\item
  What is Social Class?
\end{itemize}

\hypertarget{readings-16}{%
\subsubsection*{Readings}\label{readings-16}}
\addcontentsline{toc}{subsubsection}{Readings}

\begin{itemize}
\tightlist
\item
  Andersen, Chapter 8 - ``Social Class and Social Stratification''
\end{itemize}

\hypertarget{assignments-due-3}{%
\subsubsection*{Assignments Due}\label{assignments-due-3}}
\addcontentsline{toc}{subsubsection}{Assignments Due}

\begin{itemize}
\tightlist
\item
  Entry Ticket - Contextualizing Michael Brown (Google Classroom)
\end{itemize}

\newpage

\hypertarget{week-10}{%
\section*{Week 10}\label{week-10}}
\addcontentsline{toc}{section}{Week 10}

\hypertarget{lecture-15---tuesday-october-29th}{%
\subsection*{\texorpdfstring{Lecture 15 - Tuesday, October 29\textsuperscript{th}}{Lecture 15 - Tuesday, October 29th}}\label{lecture-15---tuesday-october-29th}}
\addcontentsline{toc}{subsection}{Lecture 15 - Tuesday, October 29\textsuperscript{th}}

\hypertarget{topics-18}{%
\subsubsection*{Topics}\label{topics-18}}
\addcontentsline{toc}{subsubsection}{Topics}

\begin{itemize}
\tightlist
\item
  Stratification \& Health
\end{itemize}

\hypertarget{readings-17}{%
\subsubsection*{Readings}\label{readings-17}}
\addcontentsline{toc}{subsubsection}{Readings}

\begin{itemize}
\tightlist
\item
  Dow, William H. and David H. Rehkopf. 2010. ``Socioeconomic gradients in health in international and historical context.'' \emph{Annals of the New York Academy of Sciences} 1186:24-36. (Electronic Reserves)
\end{itemize}

\begin{center}\rule{0.5\linewidth}{\linethickness}\end{center}

\hypertarget{exam-2---thursday-october-31st}{%
\subsection*{\texorpdfstring{Exam 2 - Thursday, October 31\textsuperscript{st}}{Exam 2 - Thursday, October 31st}}\label{exam-2---thursday-october-31st}}
\addcontentsline{toc}{subsection}{Exam 2 - Thursday, October 31\textsuperscript{st}}

\hypertarget{topics-19}{%
\subsubsection*{Topics}\label{topics-19}}
\addcontentsline{toc}{subsubsection}{Topics}

\begin{itemize}
\tightlist
\item
  covers Lectures 8 through 15, all associated readings, and the Fall Break reading on Michael Brown
\end{itemize}

\newpage

\hypertarget{week-11}{%
\section*{Week 11}\label{week-11}}
\addcontentsline{toc}{section}{Week 11}

\hypertarget{lecture-16---tuesday-november-5th}{%
\subsection*{\texorpdfstring{Lecture 16 - Tuesday, November 5\textsuperscript{th}}{Lecture 16 - Tuesday, November 5th}}\label{lecture-16---tuesday-november-5th}}
\addcontentsline{toc}{subsection}{Lecture 16 - Tuesday, November 5\textsuperscript{th}}

\hypertarget{topics-20}{%
\subsubsection*{Topics}\label{topics-20}}
\addcontentsline{toc}{subsubsection}{Topics}

\begin{itemize}
\tightlist
\item
  The Social Construction of Race
\end{itemize}

\hypertarget{readings-18}{%
\subsubsection*{Readings}\label{readings-18}}
\addcontentsline{toc}{subsubsection}{Readings}

\begin{itemize}
\tightlist
\item
  Andersen, Chapter 10 - ``Race and Ethnicity''
\item
  Cooper, Richard S., Jay S. Kaufman, and Ryk Ward. 2003. ``Race and Genomics.'' \emph{New England Journal of Medicine} 348(12):1166-1170. (Electronic Reserves)
\end{itemize}

\begin{center}\rule{0.5\linewidth}{\linethickness}\end{center}

\hypertarget{lecture-17---thursday-november-7th}{%
\subsection*{\texorpdfstring{Lecture 17 - Thursday, November 7\textsuperscript{th}}{Lecture 17 - Thursday, November 7th}}\label{lecture-17---thursday-november-7th}}
\addcontentsline{toc}{subsection}{Lecture 17 - Thursday, November 7\textsuperscript{th}}

\hypertarget{topics-21}{%
\subsubsection*{Topics}\label{topics-21}}
\addcontentsline{toc}{subsubsection}{Topics}

\begin{itemize}
\tightlist
\item
  Racial Disparities in Health
\end{itemize}

\hypertarget{readings-19}{%
\subsubsection*{Readings}\label{readings-19}}
\addcontentsline{toc}{subsubsection}{Readings}

\begin{itemize}
\tightlist
\item
  Williams, David R. and Michelle Sternthal. 2010. ``Understanding Racial-ethnic Disparities in Health : Sociological Contributions.'' \emph{Journal of Health and Social Behavior} 51(S):S15-S27. (Electronic Reserves)
\end{itemize}

\newpage

\hypertarget{week-12}{%
\section*{Week 12}\label{week-12}}
\addcontentsline{toc}{section}{Week 12}

\hypertarget{lecture-18---tuesday-november-12th}{%
\subsection*{\texorpdfstring{Lecture 18 - Tuesday, November 12\textsuperscript{th}}{Lecture 18 - Tuesday, November 12th}}\label{lecture-18---tuesday-november-12th}}
\addcontentsline{toc}{subsection}{Lecture 18 - Tuesday, November 12\textsuperscript{th}}

\hypertarget{topics-22}{%
\subsubsection*{Topics}\label{topics-22}}
\addcontentsline{toc}{subsubsection}{Topics}

\begin{itemize}
\tightlist
\item
  Gender, Sexuality, \& Health
\end{itemize}

\hypertarget{readings-20}{%
\subsubsection*{Readings}\label{readings-20}}
\addcontentsline{toc}{subsubsection}{Readings}

\begin{itemize}
\tightlist
\item
  Andersen, Chapter 11 - ``Gender''
\item
  Krieger, Nancy. 2003. ``Genders, Sexes, and Health: What Are the Connections - and Why Does It Matter?'' \emph{International Journal of Epidemiology} 32(4):652-657. (Electronic Reserves)
\end{itemize}

\begin{center}\rule{0.5\linewidth}{\linethickness}\end{center}

\hypertarget{qhq-2---thursday-november-14th}{%
\subsection*{\texorpdfstring{QHQ 2 - Thursday, November 14\textsuperscript{th}}{QHQ 2 - Thursday, November 14th}}\label{qhq-2---thursday-november-14th}}
\addcontentsline{toc}{subsection}{QHQ 2 - Thursday, November 14\textsuperscript{th}}

\hypertarget{topics-23}{%
\subsubsection*{Topics}\label{topics-23}}
\addcontentsline{toc}{subsubsection}{Topics}

\begin{itemize}
\tightlist
\item
  Discussion - \emph{Mama Might Be Better Off Dead}, Part 2
\end{itemize}

\hypertarget{readings-21}{%
\subsubsection*{Readings}\label{readings-21}}
\addcontentsline{toc}{subsubsection}{Readings}

\begin{itemize}
\tightlist
\item
  Abraham - Chapters 6 through 10
\end{itemize}

\hypertarget{assignments-due-4}{%
\subsubsection*{Assignments Due}\label{assignments-due-4}}
\addcontentsline{toc}{subsubsection}{Assignments Due}

\begin{itemize}
\tightlist
\item
  QHQ 2 (Google Classroom)

  \begin{itemize}
  \tightlist
  \item
    see the QHQ Group assignments for which chapter you should write about
  \end{itemize}
\end{itemize}

\newpage

\hypertarget{week-13}{%
\section*{Week 13}\label{week-13}}
\addcontentsline{toc}{section}{Week 13}

\hypertarget{lecture-19---tuesday-november-19th}{%
\subsection*{\texorpdfstring{Lecture 19 - Tuesday, November 19\textsuperscript{th}}{Lecture 19 - Tuesday, November 19th}}\label{lecture-19---tuesday-november-19th}}
\addcontentsline{toc}{subsection}{Lecture 19 - Tuesday, November 19\textsuperscript{th}}

\hypertarget{topics-24}{%
\subsubsection*{Topics}\label{topics-24}}
\addcontentsline{toc}{subsubsection}{Topics}

\begin{itemize}
\tightlist
\item
  The Health Care Experience
\end{itemize}

\hypertarget{readings-22}{%
\subsubsection*{Readings}\label{readings-22}}
\addcontentsline{toc}{subsubsection}{Readings}

\begin{itemize}
\tightlist
\item
  Boyer, Carol A. and Karen E. Lutfey. 2010. ``Examining Critical Health Policy Issues within and beyond the Clinical Encounter: Patient--Provider Relationships and Help-seeking Behaviors.'' \emph{Journal of Health and Social Behavior} 51(S):S80-S93. (Electronic Reserves)
\item
  Spencer, Karen Lutfey and Matthew Grace. 2016. ``Social Foundations of Health Care Inequality and Treatment Bias.'' \emph{Annual Review of Sociology} 42:101-120. (Electronic Reserves)
\end{itemize}

\begin{center}\rule{0.5\linewidth}{\linethickness}\end{center}

\hypertarget{lecture-20---thursday-november-21st}{%
\subsection*{\texorpdfstring{Lecture 20 - Thursday, November 21\textsuperscript{st}}{Lecture 20 - Thursday, November 21st}}\label{lecture-20---thursday-november-21st}}
\addcontentsline{toc}{subsection}{Lecture 20 - Thursday, November 21\textsuperscript{st}}

\hypertarget{topics-25}{%
\subsubsection*{Topics}\label{topics-25}}
\addcontentsline{toc}{subsubsection}{Topics}

\begin{itemize}
\tightlist
\item
  Indigenous Americans
\item
  Health and Native Communities
\end{itemize}

\hypertarget{documentary-2}{%
\subsubsection*{Documentary}\label{documentary-2}}
\addcontentsline{toc}{subsubsection}{Documentary}

\begin{itemize}
\tightlist
\item
  \emph{Unnatural Causes}, Part 3 - ``Bad Sugar'' (Pius Library)
\end{itemize}

\hypertarget{readings-23}{%
\subsubsection*{Readings}\label{readings-23}}
\addcontentsline{toc}{subsubsection}{Readings}

\begin{itemize}
\tightlist
\item
  Jones, David S. 2006. ``The persistence of American Indian health disparities.'' \emph{American Journal of Public Health} 96(12): 2122-2134. (Electronic Reserves)
\item
  Sarche, Michelle, and Paul Spicer. 2008. ``Poverty and health disparities for American Indian and Alaska Native children.'' \emph{Annals of the New York Academy of Sciences} 1136(1): 126-136. (Electronic Reserves)
\end{itemize}

\newpage

\hypertarget{week-14}{%
\section*{Week 14}\label{week-14}}
\addcontentsline{toc}{section}{Week 14}

\hypertarget{lecture-21---tuesday-november-26th}{%
\subsection*{\texorpdfstring{Lecture 21 - Tuesday, November 26\textsuperscript{th}}{Lecture 21 - Tuesday, November 26th}}\label{lecture-21---tuesday-november-26th}}
\addcontentsline{toc}{subsection}{Lecture 21 - Tuesday, November 26\textsuperscript{th}}

\hypertarget{topics-26}{%
\subsubsection*{Topics}\label{topics-26}}
\addcontentsline{toc}{subsubsection}{Topics}

\begin{itemize}
\tightlist
\item
  Intersectionality and Health: Race, Class, Gender, and Birth Weight
\end{itemize}

\hypertarget{documentary-3}{%
\subsubsection*{Documentary}\label{documentary-3}}
\addcontentsline{toc}{subsubsection}{Documentary}

\begin{itemize}
\tightlist
\item
  \emph{Unnatural Causes}, Part 2 - ``When the Bough Breaks'' (Pius Library)
\end{itemize}

\hypertarget{readings-24}{%
\subsubsection*{Readings}\label{readings-24}}
\addcontentsline{toc}{subsubsection}{Readings}

\begin{itemize}
\tightlist
\item
  Collins, Jr, James W. et al.~2004. ``Very Low Birthweight in African American Infants: The Role of Maternal Exposure to Interpersonal Racial Discrimination.'' \emph{American Journal of Public Health} 94(12):2132-2138. (Electronic Reserves)
\item
  David, Richard J. and James W. Collins, Jr.~1997. ``Differing Birth Weight among Infants of U.S.-Born Blacks, African-Born Blacks, and U.S.-Born Whites.'' \emph{The New England Journal of Medicine} 337:1209-1214. (Electronic Reserves)
\end{itemize}

\begin{center}\rule{0.5\linewidth}{\linethickness}\end{center}

\hypertarget{no-class---thursday-november-28th---thanksgiving-break}{%
\subsection*{\texorpdfstring{\emph{No Class} - Thursday, November 28\textsuperscript{th} - Thanksgiving Break}{No Class - Thursday, November 28th - Thanksgiving Break}}\label{no-class---thursday-november-28th---thanksgiving-break}}
\addcontentsline{toc}{subsection}{\emph{No Class} - Thursday, November 28\textsuperscript{th} - Thanksgiving Break}

\hypertarget{topics-27}{%
\subsubsection*{Topics}\label{topics-27}}
\addcontentsline{toc}{subsubsection}{Topics}

\begin{itemize}
\tightlist
\item
  Culture and Myth
\end{itemize}

\hypertarget{readings-25}{%
\subsubsection*{Readings}\label{readings-25}}
\addcontentsline{toc}{subsubsection}{Readings}

\begin{itemize}
\tightlist
\item
  Siskand, Janet. 1992. ``The Invention of Thanksgiving: A ritual of American nationality.'' \emph{Critique of Anthropology} 12(2): 167-191. (Electronic Reserves)
\end{itemize}

\newpage

\hypertarget{week-15}{%
\section*{Week 15}\label{week-15}}
\addcontentsline{toc}{section}{Week 15}

\hypertarget{qhq-3---tuesday-december-3rd}{%
\subsection*{\texorpdfstring{QHQ 3 - Tuesday, December 3\textsuperscript{rd}}{QHQ 3 - Tuesday, December 3rd}}\label{qhq-3---tuesday-december-3rd}}
\addcontentsline{toc}{subsection}{QHQ 3 - Tuesday, December 3\textsuperscript{rd}}

\hypertarget{topics-28}{%
\subsubsection*{Topics}\label{topics-28}}
\addcontentsline{toc}{subsubsection}{Topics}

\begin{itemize}
\tightlist
\item
  Discussion - \emph{Mama Might Be Better Off Dead}, Part 3
\end{itemize}

\hypertarget{readings-26}{%
\subsubsection*{Readings}\label{readings-26}}
\addcontentsline{toc}{subsubsection}{Readings}

\begin{itemize}
\tightlist
\item
  Abraham - Chapters 11 through \emph{Epilogue}
\end{itemize}

\hypertarget{assignments-due-5}{%
\subsubsection*{Assignments Due}\label{assignments-due-5}}
\addcontentsline{toc}{subsubsection}{Assignments Due}

\begin{itemize}
\tightlist
\item
  QHQ-03 (Google Classroom)

  \begin{itemize}
  \tightlist
  \item
    see the QHQ Group assignments for which chapter you should write about
  \end{itemize}
\end{itemize}

\begin{center}\rule{0.5\linewidth}{\linethickness}\end{center}

\hypertarget{lecture-22---thursday-december-5th}{%
\subsection*{\texorpdfstring{Lecture 22 - Thursday, December 5\textsuperscript{th}}{Lecture 22 - Thursday, December 5th}}\label{lecture-22---thursday-december-5th}}
\addcontentsline{toc}{subsection}{Lecture 22 - Thursday, December 5\textsuperscript{th}}

\hypertarget{topics-29}{%
\subsubsection*{Topics}\label{topics-29}}
\addcontentsline{toc}{subsubsection}{Topics}

\begin{itemize}
\tightlist
\item
  Health Care Reform
\item
  Course Conclusion
\end{itemize}

\hypertarget{readings-27}{%
\subsubsection*{Readings}\label{readings-27}}
\addcontentsline{toc}{subsubsection}{Readings}

\begin{itemize}
\tightlist
\item
  Quadagno, Jill. 2010. ``Institutions, Interest Groups, and Ideology: An Agenda for the Sociology of Health Care Reform.'' \emph{Journal of Health and Social Behavior} 51(2):125-136. (Electronic Reserves)
\item
  Williams, David. 2010. ``Beyond The Affordable Care Act: Achieving Real Improvements In Americans' Health.'' \emph{Health Affairs} 29(8):1481-1488. (Electronic Reserves)
\end{itemize}

\newpage

\hypertarget{week-16}{%
\section*{Week 16}\label{week-16}}
\addcontentsline{toc}{section}{Week 16}

\hypertarget{no-class---finals-week}{%
\subsection*{\texorpdfstring{\emph{No Class} - Finals Week}{No Class - Finals Week}}\label{no-class---finals-week}}
\addcontentsline{toc}{subsection}{\emph{No Class} - Finals Week}

\newpage

\hypertarget{week-17}{%
\section*{Week 17}\label{week-17}}
\addcontentsline{toc}{section}{Week 17}

\hypertarget{exam-3---tuesday-december-17th}{%
\subsection*{\texorpdfstring{Exam 3 - Tuesday, December 17\textsuperscript{th}}{Exam 3 - Tuesday, December 17th}}\label{exam-3---tuesday-december-17th}}
\addcontentsline{toc}{subsection}{Exam 3 - Tuesday, December 17\textsuperscript{th}}

\begin{rmdwarning}
The time for Exam 3 is different than for the rest of the course - it
will be held from 8:00am until 9:50am. The length will not be any
different than the other exams, however, so most students should be
finished prior to the official end time.
\end{rmdwarning}

\hypertarget{topics-30}{%
\subsubsection*{Topics}\label{topics-30}}
\addcontentsline{toc}{subsubsection}{Topics}

\begin{itemize}
\tightlist
\item
  covers Lectures 16 through 22, all associated readings, and the Thanksgiving Break reading
\end{itemize}

\bibliography{book.bib,packages.bib}


\end{document}
